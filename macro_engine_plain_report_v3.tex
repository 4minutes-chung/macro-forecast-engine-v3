\documentclass[11pt]{article}
\usepackage[margin=1in]{geometry}
\usepackage[hidelinks]{hyperref}
\usepackage{booktabs}
\usepackage{longtable}
\usepackage{array}
\usepackage{enumitem}
\setlength{\parindent}{0pt}
\setlength{\parskip}{0.45em}

\title{Macro Risk Engine V3: Technical Report and Freeze Record}
\author{}
\date{\today}

\begin{document}
\maketitle

\section{Executive summary}
Within this project line, V3 is the first release with explicit forecast governance.
It is designed to produce stable quarterly macro paths for PD input workflows,
with explicit acceptance gates and reproducible artifacts.

Current V3 status from the latest validation snapshot:
\begin{itemize}[leftmargin=1.2em]
  \item Release gate: \textbf{pass}
  \item Promotion gate (challenger replacing incumbent): \textbf{fail}
  \item Diagnostic-only mode: \textbf{false} (power requirements satisfied)
\end{itemize}

This means V3 is suitable to freeze as a production fallback baseline.

\section{Scope and output contract}
\subsection{Scope}
The engine forecasts macro variables. It does not estimate PD/LGD models.

\subsection{Primary PD handoff contract}
Canonical handoff is a quarterly level-path file with:
\begin{itemize}[leftmargin=1.2em]
  \item \texttt{unemployment\_rate}
  \item \texttt{ust10\_rate}
  \item \texttt{hpi\_yoy} (mapped from \texttt{hpi\_growth\_yoy})
\end{itemize}

Convenience derived outputs (for quick checks) are included but non-blocking.

\section{Data design and assumptions}
\subsection{Source and frequency}
The modeling panel is built from FRED-based series mappings and aggregated to quarterly frequency.

\subsection{Modeling panel}
The main training table is:
\begin{itemize}[leftmargin=1.2em]
  \item \texttt{data/macro\_panel\_quarterly\_model.csv}
\end{itemize}
with metadata in:
\begin{itemize}[leftmargin=1.2em]
  \item \texttt{data/macro\_panel\_metadata.json}
\end{itemize}

\subsection{Transform policy}
Rates, spreads, sentiment, and delinquency series are treated in level-style form.
Growth and inflation series are mainly YoY transforms.

\section{Model architecture from scratch}
\subsection{Horizon and regime architecture}
Forecast horizon is 80 quarters. V3 evaluates explicit regime candidates:
\begin{itemize}[leftmargin=1.2em]
  \item \textbf{Champion A (incumbent)}: Q1--Q12 short model, Q13--Q24 bridge, Q25--Q80 long-run/scenario
  \item \textbf{Champion B (challenger)}: Q1--Q16 short model, Q17--Q28 bridge, Q29--Q80 long-run/scenario
  \item \textbf{Single-stage ablation}: \texttt{single\_stage\_h12\_no\_bridge}
\end{itemize}

\subsection{Short-horizon model system}
V3 does not force one model for every variable.
Candidates include BVAR, AR, and RW. Selection is by variable and horizon bucket:
\begin{itemize}[leftmargin=1.2em]
  \item Bucket 1: Q1--Q4
  \item Bucket 2: Q5--Q12
\end{itemize}

Champion tie-break order:
\begin{enumerate}[leftmargin=1.4em]
  \item Lowest mean CRPS
  \item Coverage closeness to 0.90
  \item Width-ratio closeness to 1.0
\end{enumerate}

Champion decisions are persisted in:
\begin{itemize}[leftmargin=1.2em]
  \item \texttt{outputs/macro\_engine/champion\_map.json}
\end{itemize}

\subsection{Bridge and long-run design}
After short horizon, the system transitions by bridge dynamics and then long-run scenario behavior.
Long-run values combine structural anchors and non-structural regime parameters.

\textbf{Structural anchors (assumption sets low/base/high):}
NAIRU, inflation target, neutral real rate, productivity trend, working-age growth,
population growth, term premium.

\textbf{Non-structural long-run parameters:}
HY spread distribution behavior, mortgage spread, and housing supply drag.

\subsection{Scenario layer}
Scenarios generated in the forecast output include:
\begin{itemize}[leftmargin=1.2em]
  \item Baseline
  \item Mild\_Adverse
  \item Severe\_Adverse
  \item Demographic\_LowGrowth
\end{itemize}

\section{Validation framework (V3 governance core)}
\subsection{Contract and target set}
Validation contract version in current artifact: \texttt{v3.freeze.1}.
Primary gated target set is PD levels only:
\begin{itemize}[leftmargin=1.2em]
  \item \texttt{unemployment\_rate}
  \item \texttt{ust10\_rate}
  \item \texttt{hpi\_yoy}
\end{itemize}
across horizons h1..h12.

\subsection{Power and invariants}
\begin{itemize}[leftmargin=1.2em]
  \item Required-cell power rule: minimum \texttt{n\_oos} \(\ge 40\)
  \item Scenario timing and scenario ordering checks
  \item Boundary smoothness checks at regime seams
  \item Seeded reproducibility for simulation-based metrics
\end{itemize}

\subsection{Profiles and pass/fail usage}
\begin{itemize}[leftmargin=1.2em]
  \item \texttt{release}: production acceptance gate
  \item \texttt{promotion}: challenger replacement gate
  \item \texttt{operational}: reporting-only diagnostics
\end{itemize}

\subsection{Core metrics}
\begin{itemize}[leftmargin=1.2em]
  \item Relative RMSE vs RW (short horizon)
  \item CRPS gain vs RW (medium horizon)
  \item Coverage pass-rate (90\% interval in configured band)
  \item Width-ratio guardrails (mean and per-variable)
  \item Boundary Z metrics (median and max)
\end{itemize}

\section{Why V3 exists: issues found in earlier versions and fixes applied}
The V3 build addressed concrete failure modes observed during earlier iterations:

\small
\begin{longtable}{@{}p{0.26\textwidth}p{0.31\textwidth}p{0.37\textwidth}@{}}
\toprule
\textbf{Earlier issue} & \textbf{Risk} & \textbf{V3 response} \\
\midrule
Lag-selection mismatch between production and backtest paths & OOS claims not representative of production behavior & Unified lag-parity logic and common validation path \\
Scenario timing offset at start quarter & Scenario shocks applied one quarter off, distorting stress paths & Deterministic timing/order checks integrated in validation \\
Small OOS sample and weak evidence quality & Performance claims unstable and easy to over-interpret & Hard power gate with minimum required-cell \texttt{n\_oos} \\
Model drift from ad-hoc changes & No stable contract for release decisions & Explicit contract version, threshold profiles, and artifacts \\
Coverage failures vs over-wide intervals tradeoff & Density quality could be gamed by width inflation & Combined coverage + sharpness gates with caps and calibration controls \\
Boundary discontinuity spikes at regime seams & Artificial jumps contaminating forecast paths & Boundary continuity controls and Z-stat diagnostics \\
\bottomrule
\end{longtable}
\normalsize

\section{Latest V3 evidence snapshot}
Source: \texttt{outputs/macro\_engine/validation/validation\_summary.json}

\subsection{Release gate (incumbent)}
\begin{center}
\begin{tabular}{p{0.46\textwidth}r}
\toprule
Metric & Value \\
\midrule
Minimum required-cell \texttt{n\_oos} & 44 \\
Coverage90 pass-rate & 0.8611 \\
Mean width ratio & 1.3069 \\
Median rRMSE h1--2 & 0.9744 \\
Median rRMSE h3--4 & 0.9208 \\
Mean CRPS gain h5--12 vs RW & 15.13\% \\
Boundary median Z / max Z & 0.0012 / 0.0350 \\
Release decision & \textbf{PASS} \\
\bottomrule
\end{tabular}
\end{center}

\subsection{Promotion gate (challenger vs incumbent)}
\begin{center}
\begin{tabular}{p{0.52\textwidth}r}
\toprule
Metric & Value \\
\midrule
Challenger release pass & False \\
Promotion subset min \texttt{n\_oos} (h9..12) & 40 \\
CRPS gain h9..12 challenger vs incumbent & 1.64\% \\
Short-horizon CRPS worsen h1..4 & 0.42\% \\
Boundary comparator pass & True \\
Promotion decision & \textbf{FAIL} \\
\bottomrule
\end{tabular}
\end{center}

Interpretation: incumbent is suitable for freeze; challenger is not promoted.

\section{PD-target champion map snapshot}
From current \texttt{champion\_map.json}:
\begin{itemize}[leftmargin=1.2em]
  \item \texttt{unemployment\_rate}: BVAR in both buckets; Q5--Q12 calibration scale 1.205
  \item \texttt{ust10\_rate}: AR in both buckets; Q5--Q12 calibration scale 1.55
  \item \texttt{hpi\_growth\_yoy}: BVAR in both buckets; Q5--Q12 calibration scale 0.90
\end{itemize}

\section{Integrated scorecard and independent-review notes}
\subsection{Historical transition note (V3.1 recovery to V3 freeze)}
Earlier recovery-stage summaries recorded a period where coverage improved but width inflation still failed release checks.
That transition work introduced tiered profiles, boundary continuity enforcement, and calibration controls.
The frozen snapshot in this report reflects the post-recovery state where release now passes on the primary PD level targets.

\subsection{Detailed promotion bottleneck from scorecard diagnostics}
Promotion remains blocked for two concrete reasons in the challenger comparison:
\begin{itemize}[leftmargin=1.2em]
  \item Challenger release profile does not pass.
  \item Challenger medium-horizon CRPS gain vs incumbent on h9..h12 is 1.64\%, below the 5.0\% promotion threshold.
\end{itemize}

Short-horizon challenger degradation is controlled (0.42\% worsen on h1..h4, within threshold),
and boundary comparator conditions pass, so the blocker is model-quality gain rather than governance mechanics.

\subsection{Coverage-fail diagnostics retained}
Coverage-fail diagnostics identify localized misspecification zones rather than broad instability.
Current retained diagnostic table highlights fail cells in:
\begin{itemize}[leftmargin=1.2em]
  \item \texttt{hpi\_growth\_yoy}: h5, h6, h11
  \item \texttt{ust10\_rate}: h3, h4
\end{itemize}

\subsection{Ablation context retained}
PD-target ablation artifacts are preserved to document regime/scope tradeoffs.
The recorded winner in retained artifacts is Champion A with full variable scope,
while release-like checks on alternative candidates remain non-passing.

\subsection{Next-change priority for future branch work}
The highest-value single experiment remains challenger \texttt{ust10\_rate} density improvement in bucket Q5--Q12.
Target outcome is to reduce challenger width pressure while lifting medium-horizon CRPS gain,
so promotion thresholds can be approached without loosening governance.

\section{Shippable outputs in frozen V3}
\subsection{Canonical PD handoff}
\begin{itemize}[leftmargin=1.2em]
  \item \texttt{outputs/macro\_engine/pd\_regressors\_forecast\_levels.csv}
  \item \texttt{outputs/macro\_engine/pd\_regressors\_forecast\_derived.csv}
  \item \texttt{outputs/macro\_engine/pd\_regressors\_metadata.json}
\end{itemize}

\subsection{Governance and diagnostics}
\begin{itemize}[leftmargin=1.2em]
  \item \texttt{outputs/macro\_engine/validation/validation\_summary.json}
  \item \texttt{outputs/macro\_engine/validation/backtest\_metrics.csv}
  \item \texttt{outputs/macro\_engine/validation/calibration\_factors.json}
  \item \texttt{outputs/macro\_engine/validation/pd\_ablation\_results.csv}
  \item \texttt{outputs/macro\_engine/champion\_map.json}
\end{itemize}

\subsection{Legacy baseline table retained}
\begin{itemize}[leftmargin=1.2em]
  \item \texttt{outputs/macro\_engine/bvar\_oos\_backtest\_table.csv}
\end{itemize}

\subsection{Validation artifact retention policy in frozen package}
The frozen V3 package keeps both final-result validation artifacts and extended diagnostics.
Core files (\texttt{validation\_summary}, \texttt{backtest\_metrics}, \texttt{calibration\_factors},
\texttt{pd\_ablation\_results}) support the headline release decision.
Extended files (boundary culprits, coverage fail cells, DM table, seed registry,
scenario checks, and assumption/regime rerun folders) are retained for deeper audit trails.

\section{Limitations and next-work boundary}
\begin{itemize}[leftmargin=1.2em]
  \item This package uses revised-history macro data, not real-time vintage data. Forecast quality under true real-time information sets may differ.
  \item Promotion remains blocked. Challenger regime does not yet meet replacement criteria, mainly because medium-horizon density improvement is insufficient.
  \item The primary validation focus is the three PD level targets. This is intentional for handoff quality, but it does not imply all secondary macro variables are equally optimized.
  \item Calibration improves coverage behavior but introduces model-risk tradeoffs in width. Width controls are enforced, but further refinement may still improve sharpness efficiency.
  \item Scenario design remains rule-based and assumption-driven in long horizon. It is not a structural macroeconomic simulator and should be interpreted as disciplined scenario path generation.
  \item Derived PD deltas are convenience outputs only. Final PD feature policy, transformations, and segmentation remain the responsibility of the downstream PD modeling layer.
  \item This freeze captures one validated snapshot in time. Re-estimation on new data vintages may alter champion choices and gate outcomes.
\end{itemize}

\section{Freeze statement}
V3 is designated as the stable fallback baseline for this repository.
It is intended for reproducible reruns, controlled handoff, and governance traceability when experimental branches (V4+) are under active change.
Version progression should therefore follow a branch-and-compare workflow:
retain V3 unchanged for baseline reference, and evaluate all new modeling ideas in V4+ against this frozen benchmark before any future promotion.

\end{document}
